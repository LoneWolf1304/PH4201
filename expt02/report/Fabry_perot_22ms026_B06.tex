\documentclass[usenames, svgnames, dvipsnames]{scrartcl}
\usepackage{pgfplots}
\usepackage{emoji}
\usepackage{siunitx}
\usepackage{ gensymb }
\usepackage[makeroom]{cancel}
% \usepackage{chemformula}
\usepackage[version=3]{mhchem}
\usepackage{simpler-wick}
\newcommand{\ech}{\mathbb{e}}
\newcommand{\gcod}{\mathfrak{D}}
\newcommand{\tz}{{t\tund{0}}}
\newcommand{\php}[1]{\phi_{#1}^+}
\newcommand{\phm}[1]{\phi_{#1}^-}
\newcommand{\fp}{\triangle}
\newcommand{\pho}{\phi_1}
\newcommand{\phii}{\phi_2}
\newcommand{\phiii}{\phi_3}
\newcommand{\phiv}{\phi_4}
\newcommand{\phv}{\phi_5}
\newcommand{\phvi}{\phi_6}
\newcommand{\ti}{{t\tund{1}}}
\newcommand{\tii}{{t\tund{2}}}
\newcommand{\tiii}{{t\tund{3}}}
\newcommand{\tiv}{{t\tund{4}}}
\newcommand{\tv}{{t\tund{5}}}
\usepackage{mathbbol}
\newcommand{\iqe}{iq\ech}
\newcommand{\vph}{\varphi}
\newcommand{\MH}{\mathrm{H}}
\newcommand{\ut}{\uptau}
\newcommand{\phint}{\phi\tund{I}}
\usepackage{simpler-wick}
\usepackage{mathrsfs}
\usepackage{mathtools}
\usepackage{booktabs} % For better table formatting
\usepackage{calc}
\usepackage{Style_File}
\newcommand{\ad}[1]{a_{#1}^\dagger}
\usepackage{fancyhdr}
\newcommand{\pdx}{\vb{p}\cdot \vb{x}}
\newcommand{\creap}{{\hat{a}_{\vb{p}}}^{s\dagger}}
\newcommand{\annap}{{\hat{a}_{\vb{p}}}^{s}}
\newcommand{\crebp}{{\hat{b}_{\vb{p}}}^{s\dagger}}
\newcommand{\annbp}{{\hat{b}_{\vb{p}}}^{s}}


\newcommand{\creapr}{{\hat{a}_{\vb{p}}}^{r\dagger}}
\newcommand{\annapr}{{\hat{a}_{\vb{p}}}^{r}}
\newcommand{\crebpr}{{\hat{b}_{\vb{p}}}^{r\dagger}}
\newcommand{\annbpr}{{\hat{b}_{\vb{p}}}^{r}}
\newcommand{\creampr}{{\hat{a}_{-\vb{p}}}^{r\dagger}}
\newcommand{\annampr}{{\hat{a}_{-\vb{p}}}^{r}}
\newcommand{\crebmpr}{{\hat{b}_{-\vb{p}}}^{r\dagger}}
\newcommand{\annbmpr}{{\hat{b}_{-\vb{p}}}^{r}}
\newcommand{\acomtr}[2]{\{#1,#2\}}
\newcommand{\pdy}{\vb{p}\cdot \vb{y}}
\newcommand{\qdy}{\vb{q}\cdot \vb{y}}

\newcommand{\qdx}{\vb{q}\cdot \vb{x}}
\newcommand{\creaq}{{\hat{a}_{\vb{q}}}^{s\dagger}}
\newcommand{\annaq}{{\hat{a}_{\vb{q}}}^{s}}
\newcommand{\crebq}{{\hat{b}_{\vb{q}}}^{s\dagger}}
\newcommand{\annbq}{{\hat{b}_{\vb{q}}}^{s}}

\newcommand{\creaqr}{{\hat{a}_{\vb{q}}}^{r\dagger}}
\newcommand{\annaqr}{{\hat{a}_{\vb{q}}}^{r}}
\newcommand{\crebqr}{{\hat{b}_{\vb{q}}}^{r\dagger}}
\newcommand{\annbqr}{{\hat{b}_{\vb{q}}}^{r}}

\newcommand{\creamqr}{{\hat{a}_{-\vb{q}}}^{r\dagger}}
\newcommand{\annamqr}{{\hat{a}_{-\vb{q}}}^{r}}
\newcommand{\crebmqr}{{\hat{b}_{-\vb{q}}}^{r\dagger}}
\newcommand{\annbmqr}{{\hat{b}_{-\vb{q}}}^{r}}

% \newcommand{\crebp}{def}
\newenvironment{bMatrix}[1]{%
\bmatrix\array{#1}\hspace*{-0.5\arraycolsep}}%
{\endarray\endbmatrix}
\newcommand{\tund}[1]{_{\text{\tiny #1}}}
\newcommand{\tup}[1]{^{\text{\tiny #1}}}
\newcommand{\uep}{\upepsilon}
\newcommand{\sbo}[1]{\scalebox{0.56}{#1}}
\newcommand{\gsl}[1]{\cancel{#1}}
\usepackage{tensor}
\usepackage{array}
\usepackage{lmodern}
\definecolor{violet}{rgb}{0.5,0.27,0.45}
\newcommand{\hint}{\MH\tund{I}}
\newcommand{\fvec}[1]{\vb{#1}}
\newcommand{\cre}[1]{\hat{a}_{#1}^\dagger}
\newcommand{\ann}[1]{\hat{a}_{#1}}

\newcommand{\crep}{\hat{a}_{\vb{p}}^\dagger}
\newcommand{\annp}{\hat{a}_{\vb{p}}}
\newcommand{\fo}{\hat{\phi}}
\newcommand{\crepd}{\hat{a}_{\vb{p}'}^\dagger}
\newcommand{\annpd}{\hat{a}_{\vb{p}'}}
\newcommand{\vac}{\ket{\Omega}}
\newcommand{\kb}{k\tund{B}}
\newcommand{\vacb}{\bra{\Omega}}
\newcommand{\cremp}{\hat{a}_{\vb{-p}}^\dagger}
\newcommand{\annmp}{\hat{a}_{\vb{-p}}}

\newcommand{\crempd}{\hat{a}_{\vb{-p}'}^\dagger}
\newcommand{\annmpd}{\hat{a}_{\vb{-p}'}}

\newcommand{\tvec}[1]{\vb{{#1}}}
\renewcommand{\aa}[1]{a_{#1}}
\newcommand{\overbar}[1]{\mkern 1.5mu\overline{\mkern-1.5mu{#1}\mkern-1.5mu}\mkern 1.5mu}
\newcommand{\dadj}[1]{\overbar{#1}}
\newcommand{\bili}[1]{\dadj{\Psi}\ #1 \ \Psi}
\newcommand{\bilin}[3]{\dadj{\Psi}_{#1}\ #3 \ \Psi_{#2}}
\newcommand{\lrep}{\mathrm{S}[\Lambda]}
\newcommand{\lrepd}{\mathrm{S}^\dagger[\Lambda]}
\newcommand{\lrepinv}{\mathrm{S}^{-1}[\Lambda]}
\newcolumntype{P}[1]{>{\centering\arraybackslash}p{#1}}
% Recommended preamble:
\usetikzlibrary{arrows.meta}
\usetikzlibrary{backgrounds}
\usepgfplotslibrary{patchplots}
\usepgfplotslibrary{fillbetween}
\pgfplotsset{%
    layers/standard/.define layer set={%
        background,axis background,axis grid,axis ticks,axis lines,axis tick labels,pre main,main,axis descriptions,axis foreground%
    }{
        grid style={/pgfplots/on layer=axis grid},%
        tick style={/pgfplots/on layer=axis ticks},%
        axis line style={/pgfplots/on layer=axis lines},%
        label style={/pgfplots/on layer=axis descriptions},%
        legend style={/pgfplots/on layer=axis descriptions},%
        title style={/pgfplots/on layer=axis descriptions},%
        colorbar style={/pgfplots/on layer=axis descriptions},%
        ticklabel style={/pgfplots/on layer=axis tick labels},%
        axis background@ style={/pgfplots/on layer=axis background},%
        3d box foreground style={/pgfplots/on layer=axis foreground},%
    },
}
\newcommand{\amstitle}[1]{%
  {\Large\MakeUppercase{#1}}% First letter large, rest small caps (simplified)
  % OR for true small caps:
  % {\Large\MakeUppercase{\expandafter\@firstofone#1}\scshape\MakeLowercase{#1}}%
}
\usepackage[left = 0.8in,
right = 0.6in,
bottom = 0.8in,
top = 0.8in,
a4paper]{geometry}

\lstdefinelanguage{Julia}{
  morekeywords={
    abstract, break, case, catch, const, continue, do, else, elseif,
    end, export, false, for, function, global, if, import, in, let,
    local, macro, module, mutable, new, quote, return, true, try,
    type, using, while, where, struct
  },
  sensitive=true,
  morecomment=[l]\#,
  morestring=[b]",
}
\usepackage{listings}
\lstset{
  language=Julia,
  basicstyle=\ttfamily\footnotesize,
  keywordstyle=\color{blue}\bfseries,
  commentstyle=\color{gray},
  stringstyle=\color{red},
  breaklines=true,
  breakatwhitespace=true,
  showstringspaces=false,
  frame=single,
  columns=fullflexible,
  captionpos=b
}

\fancyhead[L,C]{}
\fancyhead[R]{Lab Report}
\usepackage[hidelinks]{hyperref}
\hypersetup{colorlinks=true,linkcolor=cyan!80!black, citecolor=YellowOrange,urlcolor=cyan!80!black}
\usepackage{doi}
\fancyhead[L]{ PH4201}
\fancyhead[C]{Adv. Optics Lab}
\fancyfoot[C]{\thepage}
\fancyfoot[R,L]{}
\usetikzlibrary{calc}
\pagestyle{fancy}
\newcommand{\delt}[2]{\tensor{\delta}{^{#1}_{#2}}}
\newcommand{\vact}{\ket{\widetilde{\Omega}}}
\newcommand{\vactb}{\bra{\widetilde{\Omega}}}
\newcommand{\meas}[1]{\frac{\dd^3\tvec{#1}}{(2\pi)^3}}
\newcommand{\tvp}{\tvec{p}}
\newcommand{\tvpd}{\tvec{p}'}
\newcommand{\ddelt}[1]{\delta^{(3)}(#1)}
\newcommand{\myeq}[1]{\stackrel{\mathclap{\tiny\mbox{#1}}}{=}}
\newcommand{\omp}{\omega_{\tvp}}
\newcommand{\ompd}{\omega_{\tvpd}}
\renewcommand{\headrulewidth}{0.4pt}
\definecolor{titleblue}{RGB}{0, 80, 120}
\usepackage{longtable} 

\renewcommand{\fdv}[2]{\frac{\delta #1}{\delta #2}}
\begin{document}
        \begin{center}
            {\rmfamily
                \Large{\textcolor{blue!30!black}{%
                    \textmd{{\Large{{F}}}}abry-\textmd{{\Large{{P}}}}erot \textmd{{\Large{{I}}}}nterferometry  \\
                     {\Large Experiment} 03%
                }}\\[0.4cm]
            }
            \end{center}
        \begin{center}
           \large \texttt{{{\LARGE  S}agnik {\LARGE S}eth\hspace{0.3cm} 22MS026 \hspace{0.3cm} Group:B6}}
        \end{center}
\hrule
\vspace{0.08cm}
\hrule
\vspace{0.6cm}
\section{Aim}
Using Fabry-Perot interferometer and simple optical elements viz. polarisers and waveplates to observe the manifestation of Pancharatnam-Berry phase through the shifting of fringe patterns. 
\section{Apparatus}
The required apparatus for performing the experiment are:
\begin{multicols}{2}
\begin{enumerate}
  \item Laser source
  \item Fabry-Perot etalon 
  \item Power meter 
  \item Focusing lens
  \item Screen
  \item Optical table
\end{enumerate}
\end{multicols}
\section{Theory}
\begin{figure}[H]
  \centering 
  

% Pattern Info
 
\tikzset{
pattern size/.store in=\mcSize, 
pattern size = 5pt,
pattern thickness/.store in=\mcThickness, 
pattern thickness = 0.3pt,
pattern radius/.store in=\mcRadius, 
pattern radius = 1pt}
\makeatletter
\pgfutil@ifundefined{pgf@pattern@name@_twufqzfan}{
\pgfdeclarepatternformonly[\mcThickness,\mcSize]{_twufqzfan}
{\pgfqpoint{0pt}{0pt}}
{\pgfpoint{\mcSize+\mcThickness}{\mcSize+\mcThickness}}
{\pgfpoint{\mcSize}{\mcSize}}
{
\pgfsetcolor{\tikz@pattern@color}
\pgfsetlinewidth{\mcThickness}
\pgfpathmoveto{\pgfqpoint{0pt}{0pt}}
\pgfpathlineto{\pgfpoint{\mcSize+\mcThickness}{\mcSize+\mcThickness}}
\pgfusepath{stroke}
}}
\makeatother

% Gradient Info
  
\tikzset {_dxqkoqpuo/.code = {\pgfsetadditionalshadetransform{ \pgftransformshift{\pgfpoint{0 bp } { 0 bp }  }  \pgftransformrotate{0 }  \pgftransformscale{2 }  }}}
\pgfdeclarehorizontalshading{_egreufrhb}{150bp}{rgb(0bp)=(0.96,0.96,0.96);
rgb(37.5bp)=(0.96,0.96,0.96);
rgb(43.821427481515066bp)=(0.86,0.86,0.89);
rgb(57.5bp)=(0.87,0.87,0.89);
rgb(62.5bp)=(0.96,0.96,0.96);
rgb(100bp)=(0.96,0.96,0.96)}
\tikzset{every picture/.style={line width=0.75pt}} %set default line width to 0.75pt        

\begin{tikzpicture}[x=0.75pt,y=0.75pt,yscale=-1,xscale=1]
%uncomment if require: \path (0,300); %set diagram left start at 0, and has height of 300

%Shape: Rectangle [id:dp8328855129785867] 
\draw  [fill={rgb, 255:red, 255; green, 0; blue, 0 }  ,fill opacity=0.9 ] (117,91.37) -- (123,91.37) -- (123,119.37) -- (117,119.37) -- cycle ;
%Shape: Path Data [id:dp20091581139839332] 
\draw  [fill={rgb, 255:red, 248; green, 235; blue, 81 }  ,fill opacity=0.67 ] (340,156) .. controls (339.14,154.26) and (338.35,152.07) .. (337.64,149.53) .. controls (334.27,140.72) and (332,123.87) .. (332,104.53) .. controls (332,86.96) and (333.88,71.44) .. (336.74,62.14) .. controls (337.66,58.47) and (338.74,55.36) .. (340,53) -- (355,53) -- (355,155.75) -- (340,155.75) -- (340,156) -- cycle ;
%Shape: Path Data [id:dp7177299008990078] 
\draw  [fill={rgb, 255:red, 248; green, 235; blue, 81 }  ,fill opacity=0.67 ] (355,53) -- (355,155.75) -- (340,155.75) -- (340,154.13) .. controls (339.66,153.68) and (339.33,153.15) .. (339,152.55) .. controls (334.9,145.06) and (332,126.38) .. (332,104.53) .. controls (332,80.94) and (335.38,61.06) .. (340,54.94) -- (340,53) -- (355,53) -- cycle ;

%Shape: Rectangle [id:dp6806938721428535] 
\draw  [pattern=_twufqzfan,pattern size=11.25pt,pattern thickness=0.75pt,pattern radius=0pt, pattern color={rgb, 255:red, 0; green, 0; blue, 0}] (447,32.95) -- (453,32.95) -- (453,175.47) -- (447,175.47) -- cycle ;
%Shape: Path Data [id:dp33620928543711626] 
\path  [shading=_egreufrhb,_dxqkoqpuo] (243.03,43.21) -- (249.64,43.21) -- (249.78,49.01) -- (288.23,49.01) -- (288.36,43.21) -- (294.97,43.21) -- (302,166.86) -- (285.47,166.86) -- (285.61,161.09) -- (252.38,161.09) -- (252.52,166.86) -- (236,166.86) -- (243.03,43.21) -- cycle (252.24,155.01) -- (285.75,155.01) -- (288.07,55.54) -- (249.93,55.54) -- (252.24,155.01) -- cycle ; % for fading 
 \draw   (243.03,43.21) -- (249.64,43.21) -- (249.78,49.01) -- (288.23,49.01) -- (288.36,43.21) -- (294.97,43.21) -- (302,166.86) -- (285.47,166.86) -- (285.61,161.09) -- (252.38,161.09) -- (252.52,166.86) -- (236,166.86) -- (243.03,43.21) -- cycle (252.24,155.01) -- (285.75,155.01) -- (288.07,55.54) -- (249.93,55.54) -- (252.24,155.01) -- cycle ; % for border 

%Straight Lines [id:da8412820819349593] 
\draw    (123.25,114.03) -- (189.18,143.73) ;
\draw [shift={(191,144.55)}, rotate = 204.25] [color={rgb, 255:red, 0; green, 0; blue, 0 }  ][line width=0.75]    (6.56,-1.97) .. controls (4.17,-0.84) and (1.99,-0.18) .. (0,0) .. controls (1.99,0.18) and (4.17,0.84) .. (6.56,1.97)   ;
%Straight Lines [id:da3791107614020697] 
\draw    (198,145) -- (236.01,140.36) ;
\draw [shift={(238,140.12)}, rotate = 173.04] [color={rgb, 255:red, 0; green, 0; blue, 0 }  ][line width=0.75]    (6.56,-1.97) .. controls (4.17,-0.84) and (1.99,-0.18) .. (0,0) .. controls (1.99,0.18) and (4.17,0.84) .. (6.56,1.97)   ;
%Straight Lines [id:da9637509017276263] 
\draw    (283,131) -- (257.94,124.84) ;
\draw [shift={(256,124.37)}, rotate = 13.8] [color={rgb, 255:red, 0; green, 0; blue, 0 }  ][line width=0.75]    (6.56,-1.97) .. controls (4.17,-0.84) and (1.99,-0.18) .. (0,0) .. controls (1.99,0.18) and (4.17,0.84) .. (6.56,1.97)   ;
%Shape: Path Data [id:dp3529603502503247] 
\draw  [fill={rgb, 255:red, 74; green, 74; blue, 74 }  ,fill opacity=1 ] (255,62.72) -- (255,149.01) -- (252.15,149.01) -- (250.14,62.72) -- (255,62.72) -- cycle ;
%Shape: Path Data [id:dp8486313960010292] 
\draw  [fill={rgb, 255:red, 74; green, 74; blue, 74 }  ,fill opacity=1 ] (283.14,62.72) -- (283.14,149.01) -- (285.99,149.01) -- (288,62.72) -- (283.14,62.72) -- cycle ;
%Straight Lines [id:da9316511896727562] 
\draw    (255,76) -- (279.07,69.5) ;
\draw [shift={(281,68.98)}, rotate = 164.9] [color={rgb, 255:red, 0; green, 0; blue, 0 }  ][line width=0.75]    (6.56,-1.97) .. controls (4.17,-0.84) and (1.99,-0.18) .. (0,0) .. controls (1.99,0.18) and (4.17,0.84) .. (6.56,1.97)   ;
%Straight Lines [id:da8142022442432358] 
\draw    (241,68) -- (221.63,54.15) ;
\draw [shift={(220,52.98)}, rotate = 35.57] [color={rgb, 255:red, 0; green, 0; blue, 0 }  ][line width=0.75]    (6.56,-1.97) .. controls (4.17,-0.84) and (1.99,-0.18) .. (0,0) .. controls (1.99,0.18) and (4.17,0.84) .. (6.56,1.97)   ;
%Straight Lines [id:da6674282070139261] 
\draw    (240,83.65) -- (221.59,69.69) ;
\draw [shift={(220,68.48)}, rotate = 37.17] [color={rgb, 255:red, 0; green, 0; blue, 0 }  ][line width=0.75]    (6.56,-1.97) .. controls (4.17,-0.84) and (1.99,-0.18) .. (0,0) .. controls (1.99,0.18) and (4.17,0.84) .. (6.56,1.97)   ;
%Straight Lines [id:da4633295750425087] 
\draw    (239,99.65) -- (220.59,85.69) ;
\draw [shift={(219,84.48)}, rotate = 37.17] [color={rgb, 255:red, 0; green, 0; blue, 0 }  ][line width=0.75]    (6.56,-1.97) .. controls (4.17,-0.84) and (1.99,-0.18) .. (0,0) .. controls (1.99,0.18) and (4.17,0.84) .. (6.56,1.97)   ;
%Straight Lines [id:da22916627040418092] 
\draw    (239,115.65) -- (220.59,101.69) ;
\draw [shift={(219,100.48)}, rotate = 37.17] [color={rgb, 255:red, 0; green, 0; blue, 0 }  ][line width=0.75]    (6.56,-1.97) .. controls (4.17,-0.84) and (1.99,-0.18) .. (0,0) .. controls (1.99,0.18) and (4.17,0.84) .. (6.56,1.97)   ;
%Straight Lines [id:da8305760719598647] 
\draw    (238,134) -- (219.59,120.04) ;
\draw [shift={(218,118.83)}, rotate = 37.17] [color={rgb, 255:red, 0; green, 0; blue, 0 }  ][line width=0.75]    (6.56,-1.97) .. controls (4.17,-0.84) and (1.99,-0.18) .. (0,0) .. controls (1.99,0.18) and (4.17,0.84) .. (6.56,1.97)   ;
%Straight Lines [id:da22460048854568537] 
\draw    (281,68.98) -- (337.04,57.35) ;
\draw [shift={(339,56.94)}, rotate = 168.27] [color={rgb, 255:red, 0; green, 0; blue, 0 }  ][line width=0.75]    (6.56,-1.97) .. controls (4.17,-0.84) and (1.99,-0.18) .. (0,0) .. controls (1.99,0.18) and (4.17,0.84) .. (6.56,1.97)   ;
%Straight Lines [id:da11775483908577633] 
\draw    (282,82.52) -- (332.03,73.67) ;
\draw [shift={(334,73.32)}, rotate = 169.97] [color={rgb, 255:red, 0; green, 0; blue, 0 }  ][line width=0.75]    (6.56,-1.97) .. controls (4.17,-0.84) and (1.99,-0.18) .. (0,0) .. controls (1.99,0.18) and (4.17,0.84) .. (6.56,1.97)   ;
%Straight Lines [id:da2420787975118045] 
\draw    (283,98.65) -- (331.03,89.68) ;
\draw [shift={(333,89.32)}, rotate = 169.43] [color={rgb, 255:red, 0; green, 0; blue, 0 }  ][line width=0.75]    (6.56,-1.97) .. controls (4.17,-0.84) and (1.99,-0.18) .. (0,0) .. controls (1.99,0.18) and (4.17,0.84) .. (6.56,1.97)   ;
%Straight Lines [id:da6587652898057387] 
\draw    (283,114.65) -- (330.03,105.9) ;
\draw [shift={(332,105.53)}, rotate = 169.46] [color={rgb, 255:red, 0; green, 0; blue, 0 }  ][line width=0.75]    (6.56,-1.97) .. controls (4.17,-0.84) and (1.99,-0.18) .. (0,0) .. controls (1.99,0.18) and (4.17,0.84) .. (6.56,1.97)   ;
%Straight Lines [id:da9849674210877271] 
\draw    (300,127.22) -- (331.04,120.93) ;
\draw [shift={(333,120.53)}, rotate = 168.55] [color={rgb, 255:red, 0; green, 0; blue, 0 }  ][line width=0.75]    (6.56,-1.97) .. controls (4.17,-0.84) and (1.99,-0.18) .. (0,0) .. controls (1.99,0.18) and (4.17,0.84) .. (6.56,1.97)   ;
%Straight Lines [id:da4284319215960505] 
\draw    (355,59.22) -- (446,81.37) ;
%Straight Lines [id:da7827927034228583] 
\draw    (355,116.37) -- (446,81.37) ;
%Straight Lines [id:da6631185675556424] 
\draw    (355,72.37) -- (446,81.37) ;
%Straight Lines [id:da10625579214683667] 
\draw    (354,87.37) -- (446,81.37) ;
%Straight Lines [id:da3308735068532389] 
\draw    (354,102.37) -- (446,81.37) ;
%Straight Lines [id:da5406227117898323] 
\draw    (255,76) -- (282,82.52) ;
%Straight Lines [id:da7109832972019617] 
\draw    (255,92) -- (283,98.65) ;
%Straight Lines [id:da006596316607101804] 
\draw    (255,108) -- (283,114.65) ;
%Straight Lines [id:da036772992717800546] 
\draw    (283,114.65) -- (256,124.37) ;
%Straight Lines [id:da09247680704821226] 
\draw    (283,98.65) -- (255,108) ;
%Straight Lines [id:da20200604101055863] 
\draw    (282,82.52) -- (255,92) ;
\draw   (376.79,60.59) -- (380.19,64.58) -- (375.18,66.13) ;
\draw   (375.39,71.29) -- (378.7,74.95) -- (374.49,77.52) ;
\draw   (372.95,82.51) -- (376.72,85.7) -- (372.89,88.81) ;
\draw   (372.05,95.27) -- (376.63,97.1) -- (373.98,101.26) ;
\draw   (375.05,105.26) -- (379.63,107.1) -- (376.97,111.26) ;
%Straight Lines [id:da3518968314154667] 
\draw    (123,110) -- (187.57,127.3) ;
\draw [shift={(189.5,127.82)}, rotate = 195] [color={rgb, 255:red, 0; green, 0; blue, 0 }  ][line width=0.75]    (6.56,-1.97) .. controls (4.17,-0.84) and (1.99,-0.18) .. (0,0) .. controls (1.99,0.18) and (4.17,0.84) .. (6.56,1.97)   ;
%Straight Lines [id:da91428325264046] 
\draw [color={rgb, 255:red, 74; green, 74; blue, 74 }  ,draw opacity=0.87 ] [dash pattern={on 4.5pt off 4.5pt}]  (107,104) -- (470,104.58) ;
%Straight Lines [id:da5908901825678137] 
\draw    (255,176.58) -- (283,176.58) ;
\draw [shift={(283,176.58)}, rotate = 180] [color={rgb, 255:red, 0; green, 0; blue, 0 }  ][line width=0.75]    (0,3.91) -- (0,-3.91)(7.65,-3.43) .. controls (4.86,-1.61) and (2.31,-0.47) .. (0,0) .. controls (2.31,0.47) and (4.86,1.61) .. (7.65,3.43)   ;
\draw [shift={(255,176.58)}, rotate = 0] [color={rgb, 255:red, 0; green, 0; blue, 0 }  ][line width=0.75]    (0,3.91) -- (0,-3.91)(7.65,-3.43) .. controls (4.86,-1.61) and (2.31,-0.47) .. (0,0) .. controls (2.31,0.47) and (4.86,1.61) .. (7.65,3.43)   ;
%Straight Lines [id:da9049577468314097] 
\draw    (283,131) -- (300,127.22) ;
%Straight Lines [id:da22061796162957015] 
\draw    (255,139.2) -- (283,131) ;
%Straight Lines [id:da40436729878837463] 
\draw    (237,146.2) -- (255,139.2) ;
%Straight Lines [id:da1581018273889172] 
\draw    (238,134) -- (252,140.2) ;
%Straight Lines [id:da7668471632213285] 
\draw    (239,115.65) -- (256,124.37) ;
%Straight Lines [id:da7309466395837563] 
\draw    (239,99.65) -- (256,108.37) ;
%Straight Lines [id:da33903646329786086] 
\draw    (240,83.65) -- (256,92.37) ;
%Straight Lines [id:da029186159080042517] 
\draw    (241,67.88) -- (255,76) ;
%Shape: Path Data [id:dp03167430881467992] 
\draw  [color={rgb, 255:red, 0; green, 0; blue, 0 }  ,draw opacity=1 ][fill={rgb, 255:red, 80; green, 227; blue, 194 }  ,fill opacity=0.64 ] (193.73,162.51) -- (193.73,162.33) -- (193.63,161.42) .. controls (193.62,161.4) and (193.62,161.38) .. (193.61,161.36) .. controls (193.54,161.6) and (193.47,161.84) .. (193.4,162.08) -- (193.42,160.63) .. controls (192.2,155.77) and (191.24,149.88) .. (190.53,143.37) .. controls (189.89,140.2) and (189.5,135.75) .. (189.5,130.82) .. controls (189.5,129.4) and (189.53,128.03) .. (189.59,126.71) .. controls (189.34,120.96) and (189.24,115.03) .. (189.28,109.13) .. controls (189.22,100.8) and (189.45,92.42) .. (189.98,84.56) -- (189.86,84.56) .. controls (190.52,73.97) and (191.71,64.24) .. (193.48,56.77) -- (193.71,54.79) -- (193.7,54.3) .. controls (193.72,54.35) and (193.73,54.4) .. (193.75,54.45) -- (194.91,44.44) -- (196.38,57.15) .. controls (198.06,64.01) and (199.17,73.68) .. (199.7,84.56) -- (199.65,84.56) .. controls (200.08,92.29) and (200.21,100.69) .. (200.04,109.13) .. controls (200.46,130.58) and (198.98,151.69) .. (195.6,162.57) -- (195.59,162.12) -- (194.66,170.57) -- (193.74,162.46) .. controls (193.74,162.47) and (193.73,162.49) .. (193.73,162.51) -- cycle ;

% Text Node
\draw (262,181.4) node [anchor=north west][inner sep=0.75pt]    {$d$};
% Text Node
\draw (182,180) node [anchor=north west][inner sep=0.75pt]  [font=\scriptsize] [align=left] {Lens};
% Text Node
\draw (253,23) node [anchor=north west][inner sep=0.75pt]  [font=\scriptsize] [align=left] {Etalon};
% Text Node
\draw (322,162) node [anchor=north west][inner sep=0.75pt]  [font=\scriptsize] [align=left] {\begin{minipage}[lt]{31.68pt}\setlength\topsep{0pt}
Focusing
\begin{center}
 lens
\end{center}

\end{minipage}};
% Text Node
\draw (431,183) node [anchor=north west][inner sep=0.75pt]  [font=\scriptsize] [align=left] {Screen};
% Text Node
\draw (91,127) node [anchor=north west][inner sep=0.75pt]  [font=\scriptsize] [align=left] {\begin{minipage}[lt]{39.21pt}\setlength\topsep{0pt}
\begin{center}
Extended \\light source
\end{center}

\end{minipage}};


\end{tikzpicture}

  \caption{Schematic diagram of the Fabry-Perot experimental set-up}
\end{figure}
\section{Data and Calculations}
\section{Error Analysis}
\section{Discussion and Conclusion}
\end{document}


