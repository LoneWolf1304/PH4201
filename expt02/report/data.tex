\section{Data and Calculations}
We perform three parts in the experiment. In the first part, we obtain the distance between the two mirrors of the etalon using a known laser source. In the second part, we use this distance between the mirrors to find the wavelength of an unknown laser source. In the third part, we obtain the finesse of the fringes formed due to interference.
\subsection{Distance between Plates}
We set up the apparatus and obtained the interference fringes from the rays on a screen. We measured the outer radius of the fringes formed and then plotted it against the order of the fringe.
\begin{figure}[H]
    \centering
    \includegraphics[scale=0.6]{../distance}
\end{figure}
\noindent
We performed a linear fit of this data with $y=mx+c$ and obtained the slope and intercept respectively as,
$$m= 2.045\pm 0.011 \qquad c= -0.243\pm 0.059$$ 
\subsection{Wavelength of Unknown Laser}
\begin{figure}[H]
    \centering
    \includegraphics[scale=0.6]{../wavelength}
\end{figure}
We performed a linear fit of this data with $y=mx+c$ and obtained the slope and intercept respectively as,
$$m= 1.656\pm 0.012 \qquad c= -0.210\pm 0.07$$ 
\subsection{Finesse Calculation}