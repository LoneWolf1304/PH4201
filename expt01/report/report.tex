\documentclass[usenames, svgnames, dvipsnames]{scrartcl}
\usepackage{pgfplots}
\usepackage{emoji}
\usepackage{siunitx}
\usepackage{ gensymb }
\usepackage[makeroom]{cancel}
% \usepackage{chemformula}
\usepackage[version=3]{mhchem}
\usepackage{simpler-wick}
\newcommand{\ech}{\mathbb{e}}
\newcommand{\gcod}{\mathfrak{D}}
\newcommand{\tz}{{t\tund{0}}}
\newcommand{\php}[1]{\phi_{#1}^+}
\newcommand{\phm}[1]{\phi_{#1}^-}
\newcommand{\fp}{\triangle}
\newcommand{\pho}{\phi_1}
\newcommand{\phii}{\phi_2}
\newcommand{\phiii}{\phi_3}
\newcommand{\phiv}{\phi_4}
\newcommand{\phv}{\phi_5}
\newcommand{\phvi}{\phi_6}
\newcommand{\ti}{{t\tund{1}}}
\newcommand{\tii}{{t\tund{2}}}
\newcommand{\tiii}{{t\tund{3}}}
\newcommand{\tiv}{{t\tund{4}}}
\newcommand{\tv}{{t\tund{5}}}
\usepackage{mathbbol}
\newcommand{\iqe}{iq\ech}
\newcommand{\vph}{\varphi}
\newcommand{\MH}{\mathrm{H}}
\newcommand{\ut}{\uptau}
\newcommand{\phint}{\phi\tund{I}}
\usepackage{simpler-wick}
\usepackage{mathrsfs}
\usepackage{mathtools}
\usepackage{booktabs} % For better table formatting
\usepackage{calc}
\usepackage{Style_File}
\newcommand{\ad}[1]{a_{#1}^\dagger}
\usepackage{fancyhdr}
\newcommand{\pdx}{\vb{p}\cdot \vb{x}}
\newcommand{\creap}{{\hat{a}_{\vb{p}}}^{s\dagger}}
\newcommand{\annap}{{\hat{a}_{\vb{p}}}^{s}}
\newcommand{\crebp}{{\hat{b}_{\vb{p}}}^{s\dagger}}
\newcommand{\annbp}{{\hat{b}_{\vb{p}}}^{s}}


\newcommand{\creapr}{{\hat{a}_{\vb{p}}}^{r\dagger}}
\newcommand{\annapr}{{\hat{a}_{\vb{p}}}^{r}}
\newcommand{\crebpr}{{\hat{b}_{\vb{p}}}^{r\dagger}}
\newcommand{\annbpr}{{\hat{b}_{\vb{p}}}^{r}}
\newcommand{\creampr}{{\hat{a}_{-\vb{p}}}^{r\dagger}}
\newcommand{\annampr}{{\hat{a}_{-\vb{p}}}^{r}}
\newcommand{\crebmpr}{{\hat{b}_{-\vb{p}}}^{r\dagger}}
\newcommand{\annbmpr}{{\hat{b}_{-\vb{p}}}^{r}}
\newcommand{\acomtr}[2]{\{#1,#2\}}
\newcommand{\pdy}{\vb{p}\cdot \vb{y}}
\newcommand{\qdy}{\vb{q}\cdot \vb{y}}

\newcommand{\qdx}{\vb{q}\cdot \vb{x}}
\newcommand{\creaq}{{\hat{a}_{\vb{q}}}^{s\dagger}}
\newcommand{\annaq}{{\hat{a}_{\vb{q}}}^{s}}
\newcommand{\crebq}{{\hat{b}_{\vb{q}}}^{s\dagger}}
\newcommand{\annbq}{{\hat{b}_{\vb{q}}}^{s}}

\newcommand{\creaqr}{{\hat{a}_{\vb{q}}}^{r\dagger}}
\newcommand{\annaqr}{{\hat{a}_{\vb{q}}}^{r}}
\newcommand{\crebqr}{{\hat{b}_{\vb{q}}}^{r\dagger}}
\newcommand{\annbqr}{{\hat{b}_{\vb{q}}}^{r}}

\newcommand{\creamqr}{{\hat{a}_{-\vb{q}}}^{r\dagger}}
\newcommand{\annamqr}{{\hat{a}_{-\vb{q}}}^{r}}
\newcommand{\crebmqr}{{\hat{b}_{-\vb{q}}}^{r\dagger}}
\newcommand{\annbmqr}{{\hat{b}_{-\vb{q}}}^{r}}

% \newcommand{\crebp}{def}
\newenvironment{bMatrix}[1]{%
\bmatrix\array{#1}\hspace*{-0.5\arraycolsep}}%
{\endarray\endbmatrix}
\newcommand{\tund}[1]{_{\text{\tiny #1}}}
\newcommand{\tup}[1]{^{\text{\tiny #1}}}
\newcommand{\uep}{\upepsilon}
\newcommand{\sbo}[1]{\scalebox{0.56}{#1}}
\newcommand{\gsl}[1]{\cancel{#1}}
\usepackage{tensor}
\usepackage{array}
\usepackage{lmodern}
\definecolor{violet}{rgb}{0.5,0.27,0.45}
\newcommand{\hint}{\MH\tund{I}}
\newcommand{\fvec}[1]{\vb{#1}}
\newcommand{\cre}[1]{\hat{a}_{#1}^\dagger}
\newcommand{\ann}[1]{\hat{a}_{#1}}

\newcommand{\crep}{\hat{a}_{\vb{p}}^\dagger}
\newcommand{\annp}{\hat{a}_{\vb{p}}}
\newcommand{\fo}{\hat{\phi}}
\newcommand{\crepd}{\hat{a}_{\vb{p}'}^\dagger}
\newcommand{\annpd}{\hat{a}_{\vb{p}'}}
\newcommand{\vac}{\ket{\Omega}}
\newcommand{\kb}{k\tund{B}}
\newcommand{\vacb}{\bra{\Omega}}
\newcommand{\cremp}{\hat{a}_{\vb{-p}}^\dagger}
\newcommand{\annmp}{\hat{a}_{\vb{-p}}}

\newcommand{\crempd}{\hat{a}_{\vb{-p}'}^\dagger}
\newcommand{\annmpd}{\hat{a}_{\vb{-p}'}}

\newcommand{\tvec}[1]{\vb{{#1}}}
\renewcommand{\aa}[1]{a_{#1}}
\newcommand{\overbar}[1]{\mkern 1.5mu\overline{\mkern-1.5mu{#1}\mkern-1.5mu}\mkern 1.5mu}
\newcommand{\dadj}[1]{\overbar{#1}}
\newcommand{\bili}[1]{\dadj{\Psi}\ #1 \ \Psi}
\newcommand{\bilin}[3]{\dadj{\Psi}_{#1}\ #3 \ \Psi_{#2}}
\newcommand{\lrep}{\mathrm{S}[\Lambda]}
\newcommand{\lrepd}{\mathrm{S}^\dagger[\Lambda]}
\newcommand{\lrepinv}{\mathrm{S}^{-1}[\Lambda]}
\newcolumntype{P}[1]{>{\centering\arraybackslash}p{#1}}
% Recommended preamble:
\usetikzlibrary{arrows.meta}
\usetikzlibrary{backgrounds}
\usepgfplotslibrary{patchplots}
\usepgfplotslibrary{fillbetween}
\pgfplotsset{%
    layers/standard/.define layer set={%
        background,axis background,axis grid,axis ticks,axis lines,axis tick labels,pre main,main,axis descriptions,axis foreground%
    }{
        grid style={/pgfplots/on layer=axis grid},%
        tick style={/pgfplots/on layer=axis ticks},%
        axis line style={/pgfplots/on layer=axis lines},%
        label style={/pgfplots/on layer=axis descriptions},%
        legend style={/pgfplots/on layer=axis descriptions},%
        title style={/pgfplots/on layer=axis descriptions},%
        colorbar style={/pgfplots/on layer=axis descriptions},%
        ticklabel style={/pgfplots/on layer=axis tick labels},%
        axis background@ style={/pgfplots/on layer=axis background},%
        3d box foreground style={/pgfplots/on layer=axis foreground},%
    },
}
\newcommand{\amstitle}[1]{%
  {\Large\MakeUppercase{#1}}% First letter large, rest small caps (simplified)
  % OR for true small caps:
  % {\Large\MakeUppercase{\expandafter\@firstofone#1}\scshape\MakeLowercase{#1}}%
}
\usepackage[left = 0.8in,
right = 0.6in,
bottom = 0.8in,
top = 0.8in,
a4paper]{geometry}

\lstdefinelanguage{Julia}{
  morekeywords={
    abstract, break, case, catch, const, continue, do, else, elseif,
    end, export, false, for, function, global, if, import, in, let,
    local, macro, module, mutable, new, quote, return, true, try,
    type, using, while, where, struct
  },
  sensitive=true,
  morecomment=[l]\#,
  morestring=[b]",
}
\usepackage{listings}
\lstset{
  language=Julia,
  basicstyle=\ttfamily\footnotesize,
  keywordstyle=\color{blue}\bfseries,
  commentstyle=\color{gray},
  stringstyle=\color{red},
  breaklines=true,
  breakatwhitespace=true,
  showstringspaces=false,
  frame=single,
  columns=fullflexible,
  captionpos=b
}

\fancyhead[L,C]{}
\fancyhead[R]{Lab Report}
\usepackage[hidelinks]{hyperref}
\hypersetup{colorlinks=true,linkcolor=cyan!80!black, citecolor=YellowOrange,urlcolor=cyan!80!black}
\usepackage{doi}
\fancyhead[L]{ PH4201}
\fancyhead[C]{Adv. Optics Lab}
\fancyfoot[C]{\thepage}
\fancyfoot[R,L]{}
\usetikzlibrary{calc}
\pagestyle{fancy}
\newcommand{\delt}[2]{\tensor{\delta}{^{#1}_{#2}}}
\newcommand{\vact}{\ket{\widetilde{\Omega}}}
\newcommand{\vactb}{\bra{\widetilde{\Omega}}}
\newcommand{\meas}[1]{\frac{\dd^3\tvec{#1}}{(2\pi)^3}}
\newcommand{\tvp}{\tvec{p}}
\newcommand{\tvpd}{\tvec{p}'}
\newcommand{\ddelt}[1]{\delta^{(3)}(#1)}
\newcommand{\myeq}[1]{\stackrel{\mathclap{\tiny\mbox{#1}}}{=}}
\newcommand{\omp}{\omega_{\tvp}}
\newcommand{\ompd}{\omega_{\tvpd}}
\renewcommand{\headrulewidth}{0.4pt}
\definecolor{titleblue}{RGB}{0, 80, 120}
\usepackage{longtable} 

\renewcommand{\fdv}[2]{\frac{\delta #1}{\delta #2}}
\begin{document}
        \begin{center}
            {\rmfamily
                \Large{\textcolor{blue!30!black}{%
                    \textmd{{\Large{{O}}}}bserving \textmd{{\Large{{P}}}}ancharatnam-\textmd{{\Large{{B}}}}erry \textmd{{\Large{{P}}}}hase \textmd{{\Large{{U}}}}sing \textmd{{\Large{{I}}}}nterferometry  \\
                     {\Large Experiment} 02%
                }}\\[0.4cm]
            }
            \end{center}
        \begin{center}
           \large \texttt{{{\LARGE  S}agnik {\LARGE S}eth\hspace{0.3cm} 22MS026 \hspace{0.3cm} Group:B6}}
        \end{center}
\hrule
\vspace{0.08cm}
\hrule
\vspace{0.6cm}
\section{Aim}
Using Michelson interferometer and simple optical elements viz. polarisers and waveplates to observe the manifestation of Pancharatnam-Berry phase through the shifting of fringe patterns. 
\section{Apparatus}
The required apparatus for performing the experiment are:
\begin{multicols}{2}
\begin{enumerate}
  \item Laser source
  \item Two Quarter-Wave Plates 
  \item Two Mirrors
  \item Linear Polariser 
  \item 50:50 Beam Splitter 
  \item Magnifier lens
  \item Screen
  \item Optical table
\end{enumerate}
\end{multicols}
\section{Theory}
The experiment is based on the measurement of a geometric phase which is independent of the optical path length and is solely based on the geometry of the evolution of the electromagnetic wave. This is different from the dynamical phase $\theta\tund{d}$ due to the path length $z$ travelled by the light,
$$\theta\tund{d} = \frac{2\pi}{\lambda}\times z$$
We will discuss about \textit{Pancharatnam-Berry phase}, wherein the wave propagates in a fixed direction but undergoes a continuous change in the state of polarisation, which is represented by a closed loop in the Poincar\'e sphere.
\begin{figure}[H]
  \centering
  

% Gradient Info
  
\tikzset {_n3q0y5w30/.code = {\pgfsetadditionalshadetransform{ \pgftransformshift{\pgfpoint{0 bp } { 0 bp }  }  \pgftransformrotate{0 }  \pgftransformscale{2 }  }}}
\pgfdeclarehorizontalshading{_3az8jupjp}{150bp}{rgb(0bp)=(0.94,0.98,1);
rgb(37.5bp)=(0.94,0.98,1);
rgb(55.77827453613281bp)=(0.8,0.92,1);
rgb(62.5bp)=(0.63,0.86,1);
rgb(100bp)=(0.63,0.86,1)}
\tikzset{every picture/.style={line width=0.75pt}} %set default line width to 0.75pt        

\begin{tikzpicture}[x=0.75pt,y=0.75pt,yscale=-1,xscale=1]
%uncomment if require: \path (0,300); %set diagram left start at 0, and has height of 300

%Shape: Circle [id:dp6406212717207735] 
\path  [shading=_3az8jupjp,_n3q0y5w30] (167.03,173.32) .. controls (167.03,121.78) and (208.82,80) .. (260.36,80) .. controls (311.9,80) and (353.68,121.78) .. (353.68,173.32) .. controls (353.68,224.87) and (311.9,266.65) .. (260.36,266.65) .. controls (208.82,266.65) and (167.03,224.87) .. (167.03,173.32) -- cycle ; % for fading 
 \draw   (167.03,173.32) .. controls (167.03,121.78) and (208.82,80) .. (260.36,80) .. controls (311.9,80) and (353.68,121.78) .. (353.68,173.32) .. controls (353.68,224.87) and (311.9,266.65) .. (260.36,266.65) .. controls (208.82,266.65) and (167.03,224.87) .. (167.03,173.32) -- cycle ; % for border 

%Straight Lines [id:da5120708694986456] 
\draw    (261.59,177.05) -- (169.42,229.66) ;
\draw [shift={(167.68,230.65)}, rotate = 330.28] [color={rgb, 255:red, 0; green, 0; blue, 0 }  ][line width=0.75]    (10.93,-4.9) .. controls (6.95,-2.3) and (3.31,-0.67) .. (0,0) .. controls (3.31,0.67) and (6.95,2.3) .. (10.93,4.9)   ;
%Straight Lines [id:da15371539088200992] 
\draw    (261.59,177.05) -- (262.66,62.87) ;
\draw [shift={(262.68,60.87)}, rotate = 90.54] [color={rgb, 255:red, 0; green, 0; blue, 0 }  ][line width=0.75]    (10.93,-4.9) .. controls (6.95,-2.3) and (3.31,-0.67) .. (0,0) .. controls (3.31,0.67) and (6.95,2.3) .. (10.93,4.9)   ;
%Straight Lines [id:da4675002401704563] 
\draw    (261.59,177.05) -- (371.68,177.85) ;
\draw [shift={(373.68,177.87)}, rotate = 180.42] [color={rgb, 255:red, 0; green, 0; blue, 0 }  ][line width=0.75]    (10.93,-4.9) .. controls (6.95,-2.3) and (3.31,-0.67) .. (0,0) .. controls (3.31,0.67) and (6.95,2.3) .. (10.93,4.9)   ;
%Straight Lines [id:da08410473848441524] 
\draw [color={rgb, 255:red, 157; green, 75; blue, 220 }  ,draw opacity=1 ]   (261.59,177.05) -- (292.68,116.85) ;
%Shape: Arc [id:dp875155371210051] 
\draw  [draw opacity=0] (261.48,143.6) .. controls (267.11,143.77) and (272.42,145.21) .. (277.14,147.63) -- (260.36,180.32) -- cycle ; \draw   (261.48,143.6) .. controls (267.11,143.77) and (272.42,145.21) .. (277.14,147.63) ;  
%Shape: Arc [id:dp8442710357943789] 
\draw  [draw opacity=0] (278.92,184.97) .. controls (274.03,186.09) and (268.05,186.75) .. (261.59,186.75) .. controls (256.04,186.75) and (250.84,186.26) .. (246.38,185.41) -- (261.59,177.05) -- cycle ; \draw   (278.92,184.97) .. controls (274.03,186.09) and (268.05,186.75) .. (261.59,186.75) .. controls (256.04,186.75) and (250.84,186.26) .. (246.38,185.41) ;  
%Straight Lines [id:da31008563400391076] 
\draw  [dash pattern={on 0.84pt off 2.51pt}]  (292.68,189.4) -- (292.68,116.85) ;
%Straight Lines [id:da7200714208810066] 
\draw  [dash pattern={on 0.84pt off 2.51pt}]  (261.59,177.05) -- (292.68,189.4) ;
%Shape: Ellipse [id:dp06773549043615945] 
\draw  [fill={rgb, 255:red, 248; green, 231; blue, 28 }  ,fill opacity=0.24 ][dash pattern={on 4.5pt off 4.5pt}] (167.03,177.06) .. controls (167.03,155.27) and (208.82,137.6) .. (260.36,137.6) .. controls (311.9,137.6) and (353.68,155.27) .. (353.68,177.06) .. controls (353.68,198.85) and (311.9,216.52) .. (260.36,216.52) .. controls (208.82,216.52) and (167.03,198.85) .. (167.03,177.06) -- cycle ;
%Shape: Ellipse [id:dp24576205607243373] 
\draw  [draw opacity=0][fill={rgb, 255:red, 38; green, 12; blue, 197 }  ,fill opacity=1 ][dash pattern={on 4.5pt off 4.5pt}] (259.36,80) .. controls (259.36,78.39) and (260.66,77.08) .. (262.26,77.08) .. controls (263.87,77.08) and (265.17,78.39) .. (265.17,80) .. controls (265.17,81.61) and (263.87,82.92) .. (262.26,82.92) .. controls (260.66,82.92) and (259.36,81.61) .. (259.36,80) -- cycle ;
%Shape: Ellipse [id:dp6984956246077939] 
\draw  [draw opacity=0][fill={rgb, 255:red, 38; green, 12; blue, 197 }  ,fill opacity=1 ][dash pattern={on 4.5pt off 4.5pt}] (260.36,266.65) .. controls (260.36,265.04) and (261.66,263.73) .. (263.26,263.73) .. controls (264.87,263.73) and (266.17,265.04) .. (266.17,266.65) .. controls (266.17,268.26) and (264.87,269.57) .. (263.26,269.57) .. controls (261.66,269.57) and (260.36,268.26) .. (260.36,266.65) -- cycle ;
%Shape: Ellipse [id:dp9059789894751615] 
\draw  [draw opacity=0][fill={rgb, 255:red, 255; green, 0; blue, 33 }  ,fill opacity=1 ][dash pattern={on 4.5pt off 4.5pt}] (203.36,209.65) .. controls (203.36,208.04) and (204.66,206.73) .. (206.26,206.73) .. controls (207.87,206.73) and (209.17,208.04) .. (209.17,209.65) .. controls (209.17,211.26) and (207.87,212.57) .. (206.26,212.57) .. controls (204.66,212.57) and (203.36,211.26) .. (203.36,209.65) -- cycle ;
%Shape: Ellipse [id:dp20089683016381765] 
\draw  [draw opacity=0][fill={rgb, 255:red, 255; green, 0; blue, 33 }  ,fill opacity=1 ][dash pattern={on 4.5pt off 4.5pt}] (308.36,144.65) .. controls (308.36,143.04) and (309.66,141.73) .. (311.26,141.73) .. controls (312.87,141.73) and (314.17,143.04) .. (314.17,144.65) .. controls (314.17,146.26) and (312.87,147.57) .. (311.26,147.57) .. controls (309.66,147.57) and (308.36,146.26) .. (308.36,144.65) -- cycle ;
%Shape: Ellipse [id:dp8254418229832045] 
\draw  [draw opacity=0][fill={rgb, 255:red, 65; green, 117; blue, 5 }  ,fill opacity=1 ][dash pattern={on 4.5pt off 4.5pt}] (350.68,177.06) .. controls (350.68,175.45) and (351.98,174.14) .. (353.59,174.14) .. controls (355.19,174.14) and (356.5,175.45) .. (356.5,177.06) .. controls (356.5,178.67) and (355.19,179.98) .. (353.59,179.98) .. controls (351.98,179.98) and (350.68,178.67) .. (350.68,177.06) -- cycle ;
%Shape: Ellipse [id:dp36388662554978257] 
\draw  [draw opacity=0][fill={rgb, 255:red, 65; green, 117; blue, 5 }  ,fill opacity=1 ][dash pattern={on 4.5pt off 4.5pt}] (164.13,177.06) .. controls (164.13,175.45) and (165.43,174.14) .. (167.03,174.14) .. controls (168.64,174.14) and (169.94,175.45) .. (169.94,177.06) .. controls (169.94,178.67) and (168.64,179.98) .. (167.03,179.98) .. controls (165.43,179.98) and (164.13,178.67) .. (164.13,177.06) -- cycle ;
%Shape: Ellipse [id:dp43839158045072046] 
\draw  [draw opacity=0][fill={rgb, 255:red, 157; green, 75; blue, 220 }  ,fill opacity=1 ][dash pattern={on 4.5pt off 4.5pt}] (289.68,115.85) .. controls (289.68,114.24) and (290.98,112.93) .. (292.59,112.93) .. controls (294.19,112.93) and (295.5,114.24) .. (295.5,115.85) .. controls (295.5,117.46) and (294.19,118.77) .. (292.59,118.77) .. controls (290.98,118.77) and (289.68,117.46) .. (289.68,115.85) -- cycle ;
%Straight Lines [id:da8188010633140681] 
\draw  [dash pattern={on 0.84pt off 2.51pt}]  (261.59,177.05) -- (314.17,144.65) ;
%Straight Lines [id:da4970670972010699] 
\draw  [dash pattern={on 0.84pt off 2.51pt}]  (169.94,177.06) -- (261.59,177.05) ;
%Shape: Circle [id:dp7168582977318296] 
\draw  [color={rgb, 255:red, 74; green, 144; blue, 226 }  ,draw opacity=1 ][line width=1.5]  (251,44.5) .. controls (251,38.7) and (255.7,34) .. (261.5,34) .. controls (267.3,34) and (272,38.7) .. (272,44.5) .. controls (272,50.3) and (267.3,55) .. (261.5,55) .. controls (255.7,55) and (251,50.3) .. (251,44.5) -- cycle ;
%Straight Lines [id:da42356120410196907] 
\draw  [dash pattern={on 0.84pt off 2.51pt}]  (261.59,177.05) -- (263.26,263.73) ;
%Shape: Circle [id:dp7373946176406471] 
\draw  [color={rgb, 255:red, 74; green, 144; blue, 226 }  ,draw opacity=1 ][line width=1.5]  (254,282.5) .. controls (254,276.7) and (258.7,272) .. (264.5,272) .. controls (270.3,272) and (275,276.7) .. (275,282.5) .. controls (275,288.3) and (270.3,293) .. (264.5,293) .. controls (258.7,293) and (254,288.3) .. (254,282.5) -- cycle ;
\draw  [color={rgb, 255:red, 74; green, 144; blue, 226 }  ,draw opacity=1 ][line width=1.5]  (276.05,41.35) -- (272.15,47.11) -- (267.91,41.59) ;
\draw  [color={rgb, 255:red, 74; green, 144; blue, 226 }  ,draw opacity=1 ][line width=1.5]  (258.14,279.47) -- (254.06,285.11) -- (249.99,279.47) ;
%Straight Lines [id:da31433845457985843] 
\draw [color={rgb, 255:red, 74; green, 144; blue, 226 }  ,draw opacity=1 ][line width=1.5]    (192,220.02) -- (220,220.24) ;
\draw [shift={(223,220.27)}, rotate = 180.45] [color={rgb, 255:red, 74; green, 144; blue, 226 }  ,draw opacity=1 ][line width=1.5]    (8.53,-2.57) .. controls (5.42,-1.09) and (2.58,-0.23) .. (0,0) .. controls (2.58,0.23) and (5.42,1.09) .. (8.53,2.57)   ;
\draw [shift={(189,220)}, rotate = 0.45] [color={rgb, 255:red, 74; green, 144; blue, 226 }  ,draw opacity=1 ][line width=1.5]    (8.53,-2.57) .. controls (5.42,-1.09) and (2.58,-0.23) .. (0,0) .. controls (2.58,0.23) and (5.42,1.09) .. (8.53,2.57)   ;
%Shape: Boxed Line [id:dp7497697764799951] 
\draw [color={rgb, 255:red, 74; green, 144; blue, 226 }  ,draw opacity=1 ][line width=1.5]    (312.11,107.13) -- (311.89,135.13) ;
\draw [shift={(311.87,138.13)}, rotate = 270.45] [color={rgb, 255:red, 74; green, 144; blue, 226 }  ,draw opacity=1 ][line width=1.5]    (8.53,-2.57) .. controls (5.42,-1.09) and (2.58,-0.23) .. (0,0) .. controls (2.58,0.23) and (5.42,1.09) .. (8.53,2.57)   ;
\draw [shift={(312.13,104.13)}, rotate = 90.45] [color={rgb, 255:red, 74; green, 144; blue, 226 }  ,draw opacity=1 ][line width=1.5]    (8.53,-2.57) .. controls (5.42,-1.09) and (2.58,-0.23) .. (0,0) .. controls (2.58,0.23) and (5.42,1.09) .. (8.53,2.57)   ;
%Shape: Boxed Line [id:dp3615276405181248] 
\draw [color={rgb, 255:red, 74; green, 144; blue, 226 }  ,draw opacity=1 ][line width=1.5]    (390.98,170.31) -- (371.02,189.96) ;
\draw [shift={(368.88,192.06)}, rotate = 315.45] [color={rgb, 255:red, 74; green, 144; blue, 226 }  ,draw opacity=1 ][line width=1.5]    (8.53,-2.57) .. controls (5.42,-1.09) and (2.58,-0.23) .. (0,0) .. controls (2.58,0.23) and (5.42,1.09) .. (8.53,2.57)   ;
\draw [shift={(393.12,168.21)}, rotate = 135.45] [color={rgb, 255:red, 74; green, 144; blue, 226 }  ,draw opacity=1 ][line width=1.5]    (8.53,-2.57) .. controls (5.42,-1.09) and (2.58,-0.23) .. (0,0) .. controls (2.58,0.23) and (5.42,1.09) .. (8.53,2.57)   ;
%Shape: Boxed Line [id:dp25245506689184194] 
\draw [color={rgb, 255:red, 74; green, 144; blue, 226 }  ,draw opacity=1 ][line width=1.5]    (140.17,167.15) -- (159.83,187.12) ;
\draw [shift={(161.93,189.26)}, rotate = 225.45] [color={rgb, 255:red, 74; green, 144; blue, 226 }  ,draw opacity=1 ][line width=1.5]    (8.53,-2.57) .. controls (5.42,-1.09) and (2.58,-0.23) .. (0,0) .. controls (2.58,0.23) and (5.42,1.09) .. (8.53,2.57)   ;
\draw [shift={(138.07,165.01)}, rotate = 45.45] [color={rgb, 255:red, 74; green, 144; blue, 226 }  ,draw opacity=1 ][line width=1.5]    (8.53,-2.57) .. controls (5.42,-1.09) and (2.58,-0.23) .. (0,0) .. controls (2.58,0.23) and (5.42,1.09) .. (8.53,2.57)   ;

% Text Node
\draw (267,129) node [anchor=north west][inner sep=0.75pt]  [font=\scriptsize]  {$\theta $};
% Text Node
\draw (254,188) node [anchor=north west][inner sep=0.75pt]  [font=\scriptsize]  {$\phi $};
% Text Node
\draw (282,40) node [anchor=north west][inner sep=0.75pt]  [font=\scriptsize] [align=left] {RCP};
% Text Node
\draw (279,276) node [anchor=north west][inner sep=0.75pt]  [font=\scriptsize] [align=left] {LCP};
% Text Node
\draw (199,225) node [anchor=north west][inner sep=0.75pt]  [font=\scriptsize] [align=left] {LHP};
% Text Node
\draw (316,121) node [anchor=north west][inner sep=0.75pt]  [font=\scriptsize] [align=left] {LVP};
% Text Node
\draw (385,181) node [anchor=north west][inner sep=0.75pt]  [font=\scriptsize] [align=left] {L+45};
% Text Node
\draw (115,177) node [anchor=north west][inner sep=0.75pt]  [font=\scriptsize] [align=left] {L-45};


\end{tikzpicture}

  \caption{Poincar\'e sphere representation of polarisation states}
  \label{poincare}
\end{figure}
\noindent
The Poincar\'e sphere is a convenient representation of polarisation states of light where the Stokes vector for completely polarised light lie on the surface of the sphere and for partially polaised light lies in the interior. Change of polarisation can be depicted by rotation in the Poincar\'e sphere. The special points are shown in Fig. \ref{poincare}; the north and south pole represent the right and left circular polarised light. Along the $+x$ and $-x$ direction, we have horizontal and vertical linear polarisation. Along $+y$ and $-y$ we have the diagonal linear polarisation states.
\begin{figure}[H]
  \centering
  

% Gradient Info
  
\tikzset {_srouxavuy/.code = {\pgfsetadditionalshadetransform{ \pgftransformshift{\pgfpoint{0 bp } { 0 bp }  }  \pgftransformrotate{0 }  \pgftransformscale{2 }  }}}
\pgfdeclarehorizontalshading{_t7merp75g}{150bp}{rgb(0bp)=(0.94,0.98,1);
rgb(37.5bp)=(0.94,0.98,1);
rgb(55.77827453613281bp)=(0.8,0.92,1);
rgb(62.5bp)=(0.63,0.86,1);
rgb(100bp)=(0.63,0.86,1)}

% Pattern Info
 
\tikzset{
pattern size/.store in=\mcSize, 
pattern size = 5pt,
pattern thickness/.store in=\mcThickness, 
pattern thickness = 0.3pt,
pattern radius/.store in=\mcRadius, 
pattern radius = 1pt}
\makeatletter
\pgfutil@ifundefined{pgf@pattern@name@_9canxba34}{
\pgfdeclarepatternformonly[\mcThickness,\mcSize]{_9canxba34}
{\pgfqpoint{0pt}{0pt}}
{\pgfpoint{\mcSize+\mcThickness}{\mcSize+\mcThickness}}
{\pgfpoint{\mcSize}{\mcSize}}
{
\pgfsetcolor{\tikz@pattern@color}
\pgfsetlinewidth{\mcThickness}
\pgfpathmoveto{\pgfqpoint{0pt}{0pt}}
\pgfpathlineto{\pgfpoint{\mcSize+\mcThickness}{\mcSize+\mcThickness}}
\pgfusepath{stroke}
}}
\makeatother
\tikzset{every picture/.style={line width=0.75pt}} %set default line width to 0.75pt        

\begin{tikzpicture}[x=0.75pt,y=0.75pt,yscale=-1,xscale=1]
%uncomment if require: \path (0,305); %set diagram left start at 0, and has height of 305

%Shape: Circle [id:dp22281913455598246] 
\path  [shading=_t7merp75g,_srouxavuy] (145.03,158.32) .. controls (145.03,106.78) and (186.82,65) .. (238.36,65) .. controls (289.9,65) and (331.68,106.78) .. (331.68,158.32) .. controls (331.68,209.87) and (289.9,251.65) .. (238.36,251.65) .. controls (186.82,251.65) and (145.03,209.87) .. (145.03,158.32) -- cycle ; % for fading 
 \draw   (145.03,158.32) .. controls (145.03,106.78) and (186.82,65) .. (238.36,65) .. controls (289.9,65) and (331.68,106.78) .. (331.68,158.32) .. controls (331.68,209.87) and (289.9,251.65) .. (238.36,251.65) .. controls (186.82,251.65) and (145.03,209.87) .. (145.03,158.32) -- cycle ; % for border 

%Straight Lines [id:da27903961804041655] 
\draw    (240.59,162.05) -- (121.71,233.4) ;
\draw [shift={(120,234.43)}, rotate = 329.03] [color={rgb, 255:red, 0; green, 0; blue, 0 }  ][line width=0.75]    (10.93,-4.9) .. controls (6.95,-2.3) and (3.31,-0.67) .. (0,0) .. controls (3.31,0.67) and (6.95,2.3) .. (10.93,4.9)   ;
%Straight Lines [id:da4405460643674677] 
\draw    (240.59,162.05) -- (241.66,47.87) ;
\draw [shift={(241.68,45.87)}, rotate = 90.54] [color={rgb, 255:red, 0; green, 0; blue, 0 }  ][line width=0.75]    (10.93,-4.9) .. controls (6.95,-2.3) and (3.31,-0.67) .. (0,0) .. controls (3.31,0.67) and (6.95,2.3) .. (10.93,4.9)   ;
%Straight Lines [id:da06563433968025789] 
\draw    (240.59,162.05) -- (350.68,162.85) ;
\draw [shift={(352.68,162.87)}, rotate = 180.42] [color={rgb, 255:red, 0; green, 0; blue, 0 }  ][line width=0.75]    (10.93,-4.9) .. controls (6.95,-2.3) and (3.31,-0.67) .. (0,0) .. controls (3.31,0.67) and (6.95,2.3) .. (10.93,4.9)   ;
%Shape: Ellipse [id:dp4522869059332102] 
\draw  [draw opacity=0][fill={rgb, 255:red, 255; green, 0; blue, 33 }  ,fill opacity=1 ][dash pattern={on 4.5pt off 4.5pt}] (287.36,129.65) .. controls (287.36,128.04) and (288.66,126.73) .. (290.26,126.73) .. controls (291.87,126.73) and (293.17,128.04) .. (293.17,129.65) .. controls (293.17,131.26) and (291.87,132.57) .. (290.26,132.57) .. controls (288.66,132.57) and (287.36,131.26) .. (287.36,129.65) -- cycle ;
%Straight Lines [id:da2857252106477066] 
\draw  [dash pattern={on 0.84pt off 2.51pt}]  (240.59,162.05) -- (293.17,129.65) ;
%Straight Lines [id:da22823259421746778] 
\draw  [dash pattern={on 0.84pt off 2.51pt}]  (148.94,162.06) -- (240.59,162.05) ;
%Shape: Circle [id:dp2192257207808842] 
\draw  [color={rgb, 255:red, 74; green, 144; blue, 226 }  ,draw opacity=1 ][line width=1.5]  (230,29.5) .. controls (230,23.7) and (234.7,19) .. (240.5,19) .. controls (246.3,19) and (251,23.7) .. (251,29.5) .. controls (251,35.3) and (246.3,40) .. (240.5,40) .. controls (234.7,40) and (230,35.3) .. (230,29.5) -- cycle ;
%Straight Lines [id:da2863834819991641] 
\draw  [dash pattern={on 0.84pt off 2.51pt}]  (240.59,162.05) -- (242.26,248.73) ;
%Shape: Circle [id:dp9932878830946502] 
\draw  [color={rgb, 255:red, 74; green, 144; blue, 226 }  ,draw opacity=1 ][line width=1.5]  (233,267.5) .. controls (233,261.7) and (237.7,257) .. (243.5,257) .. controls (249.3,257) and (254,261.7) .. (254,267.5) .. controls (254,273.3) and (249.3,278) .. (243.5,278) .. controls (237.7,278) and (233,273.3) .. (233,267.5) -- cycle ;
\draw  [color={rgb, 255:red, 74; green, 144; blue, 226 }  ,draw opacity=1 ][line width=1.5]  (255.05,26.35) -- (251.15,32.11) -- (246.91,26.59) ;
\draw  [color={rgb, 255:red, 74; green, 144; blue, 226 }  ,draw opacity=1 ][line width=1.5]  (237.14,264.47) -- (233.06,270.11) -- (228.99,264.47) ;
%Straight Lines [id:da19295528395724282] 
\draw [color={rgb, 255:red, 74; green, 144; blue, 226 }  ,draw opacity=1 ][line width=1.5]    (170,205.02) -- (198,205.24) ;
\draw [shift={(201,205.27)}, rotate = 180.45] [color={rgb, 255:red, 74; green, 144; blue, 226 }  ,draw opacity=1 ][line width=1.5]    (8.53,-2.57) .. controls (5.42,-1.09) and (2.58,-0.23) .. (0,0) .. controls (2.58,0.23) and (5.42,1.09) .. (8.53,2.57)   ;
\draw [shift={(167,205)}, rotate = 0.45] [color={rgb, 255:red, 74; green, 144; blue, 226 }  ,draw opacity=1 ][line width=1.5]    (8.53,-2.57) .. controls (5.42,-1.09) and (2.58,-0.23) .. (0,0) .. controls (2.58,0.23) and (5.42,1.09) .. (8.53,2.57)   ;
%Shape: Boxed Line [id:dp6170249753949749] 
\draw [color={rgb, 255:red, 74; green, 144; blue, 226 }  ,draw opacity=1 ][line width=1.5]    (298.41,100.65) -- (298.19,128.65) ;
\draw [shift={(298.17,131.65)}, rotate = 270.45] [color={rgb, 255:red, 74; green, 144; blue, 226 }  ,draw opacity=1 ][line width=1.5]    (8.53,-2.57) .. controls (5.42,-1.09) and (2.58,-0.23) .. (0,0) .. controls (2.58,0.23) and (5.42,1.09) .. (8.53,2.57)   ;
\draw [shift={(298.44,97.65)}, rotate = 90.45] [color={rgb, 255:red, 74; green, 144; blue, 226 }  ,draw opacity=1 ][line width=1.5]    (8.53,-2.57) .. controls (5.42,-1.09) and (2.58,-0.23) .. (0,0) .. controls (2.58,0.23) and (5.42,1.09) .. (8.53,2.57)   ;
%Shape: Boxed Line [id:dp6843193009695728] 
\draw [color={rgb, 255:red, 74; green, 144; blue, 226 }  ,draw opacity=1 ][line width=1.5]    (369.98,155.31) -- (350.02,174.96) ;
\draw [shift={(347.88,177.06)}, rotate = 315.45] [color={rgb, 255:red, 74; green, 144; blue, 226 }  ,draw opacity=1 ][line width=1.5]    (8.53,-2.57) .. controls (5.42,-1.09) and (2.58,-0.23) .. (0,0) .. controls (2.58,0.23) and (5.42,1.09) .. (8.53,2.57)   ;
\draw [shift={(372.12,153.21)}, rotate = 135.45] [color={rgb, 255:red, 74; green, 144; blue, 226 }  ,draw opacity=1 ][line width=1.5]    (8.53,-2.57) .. controls (5.42,-1.09) and (2.58,-0.23) .. (0,0) .. controls (2.58,0.23) and (5.42,1.09) .. (8.53,2.57)   ;
%Shape: Boxed Line [id:dp7986971790273789] 
\draw [color={rgb, 255:red, 74; green, 144; blue, 226 }  ,draw opacity=1 ][line width=1.5]    (119.17,152.15) -- (138.83,172.12) ;
\draw [shift={(140.93,174.26)}, rotate = 225.45] [color={rgb, 255:red, 74; green, 144; blue, 226 }  ,draw opacity=1 ][line width=1.5]    (8.53,-2.57) .. controls (5.42,-1.09) and (2.58,-0.23) .. (0,0) .. controls (2.58,0.23) and (5.42,1.09) .. (8.53,2.57)   ;
\draw [shift={(117.07,150.01)}, rotate = 45.45] [color={rgb, 255:red, 74; green, 144; blue, 226 }  ,draw opacity=1 ][line width=1.5]    (8.53,-2.57) .. controls (5.42,-1.09) and (2.58,-0.23) .. (0,0) .. controls (2.58,0.23) and (5.42,1.09) .. (8.53,2.57)   ;
\draw   (185.46,114.36) -- (192.8,108.23) -- (194,117.72) ;
\draw   (210.7,211.36) -- (217.41,218.17) -- (219.53,208.84) ;
\draw   (198.34,230.05) -- (199.71,222.48) -- (206.91,225.19) ;
\draw   (207.02,118.83) -- (209.09,128.16) -- (215.84,121.39) ;
%Shape: Ellipse [id:dp11025893347051219] 
\draw  [draw opacity=0][fill={rgb, 255:red, 38; green, 12; blue, 197 }  ,fill opacity=1 ][dash pattern={on 4.5pt off 4.5pt}] (239.36,251.65) .. controls (239.36,250.04) and (240.66,248.73) .. (242.26,248.73) .. controls (243.87,248.73) and (245.17,250.04) .. (245.17,251.65) .. controls (245.17,253.26) and (243.87,254.57) .. (242.26,254.57) .. controls (240.66,254.57) and (239.36,253.26) .. (239.36,251.65) -- cycle ;
%Shape: Ellipse [id:dp6305602974557984] 
\draw  [draw opacity=0][fill={rgb, 255:red, 38; green, 12; blue, 197 }  ,fill opacity=1 ][dash pattern={on 4.5pt off 4.5pt}] (238.36,65) .. controls (238.36,63.39) and (239.66,62.08) .. (241.26,62.08) .. controls (242.87,62.08) and (244.17,63.39) .. (244.17,65) .. controls (244.17,66.61) and (242.87,67.92) .. (241.26,67.92) .. controls (239.66,67.92) and (238.36,66.61) .. (238.36,65) -- cycle ;
%Shape: Path Data [id:dp6389414974032115] 
\draw  [pattern=_9canxba34,pattern size=3.75pt,pattern thickness=0.75pt,pattern radius=0pt, pattern color={rgb, 255:red, 74; green, 144; blue, 226}] (242.26,251.35) .. controls (242.26,251.35) and (242.26,251.35) .. (242.26,251.35) -- (242.26,250.14) .. controls (242.26,250.14) and (242.26,250.14) .. (242.26,250.14) -- (242.26,251.35) -- cycle (241.26,64.78) .. controls (241.07,64.97) and (240.87,65.17) .. (240.67,65.37) .. controls (226.07,74.69) and (214.48,97.41) .. (209.56,126.27) .. controls (203.44,151.36) and (202.91,180.34) .. (213.94,208.32) .. controls (214.06,208.74) and (214.19,209.16) .. (214.32,209.57) .. controls (214.99,211.23) and (215.7,212.89) .. (216.46,214.54) .. controls (221.97,229.39) and (229.57,240.98) .. (238.39,247.6) .. controls (239.44,248.76) and (240.52,249.91) .. (241.63,251.06) .. controls (241.76,251.14) and (241.9,251.22) .. (242.03,251.3) .. controls (173.93,235.8) and (145.11,111.03) .. (241.26,64.78) -- (241.26,65) -- (241.26,65) -- (241.26,64.78) -- cycle ;
%Shape: Ellipse [id:dp5750519022743181] 
\draw  [draw opacity=0][fill={rgb, 255:red, 255; green, 0; blue, 33 }  ,fill opacity=1 ][dash pattern={on 4.5pt off 4.5pt}] (182.36,194.65) .. controls (182.36,193.04) and (183.66,191.73) .. (185.26,191.73) .. controls (186.87,191.73) and (188.17,193.04) .. (188.17,194.65) .. controls (188.17,196.26) and (186.87,197.57) .. (185.26,197.57) .. controls (183.66,197.57) and (182.36,196.26) .. (182.36,194.65) -- cycle ;
%Shape: Ellipse [id:dp5168931499824684] 
\draw  [fill={rgb, 255:red, 248; green, 231; blue, 28 }  ,fill opacity=0.24 ][dash pattern={on 4.5pt off 4.5pt}] (145.27,163.05) .. controls (145.27,141.26) and (187.05,123.59) .. (238.59,123.59) .. controls (290.13,123.59) and (331.92,141.26) .. (331.92,163.05) .. controls (331.92,184.84) and (290.13,202.51) .. (238.59,202.51) .. controls (187.05,202.51) and (145.27,184.84) .. (145.27,163.05) -- cycle ;
%Shape: Ellipse [id:dp8781645008033632] 
\draw  [draw opacity=0][fill={rgb, 255:red, 65; green, 117; blue, 5 }  ,fill opacity=1 ][dash pattern={on 4.5pt off 4.5pt}] (329.01,163.05) .. controls (329.01,161.44) and (330.31,160.13) .. (331.92,160.13) .. controls (333.52,160.13) and (334.82,161.44) .. (334.82,163.05) .. controls (334.82,164.66) and (333.52,165.97) .. (331.92,165.97) .. controls (330.31,165.97) and (329.01,164.66) .. (329.01,163.05) -- cycle ;
%Shape: Ellipse [id:dp8562802289314954] 
\draw  [draw opacity=0][fill={rgb, 255:red, 65; green, 117; blue, 5 }  ,fill opacity=1 ][dash pattern={on 4.5pt off 4.5pt}] (142.13,162.06) .. controls (142.13,160.45) and (143.43,159.14) .. (145.03,159.14) .. controls (146.64,159.14) and (147.94,160.45) .. (147.94,162.06) .. controls (147.94,163.67) and (146.64,164.98) .. (145.03,164.98) .. controls (143.43,164.98) and (142.13,163.67) .. (142.13,162.06) -- cycle ;
%Straight Lines [id:da4770534941293796] 
\draw  [dash pattern={on 4.5pt off 4.5pt}]  (240.59,162.05) -- (155.25,268.87) ;
\draw [shift={(154,270.43)}, rotate = 308.62] [color={rgb, 255:red, 0; green, 0; blue, 0 }  ][line width=0.75]    (6.56,-1.97) .. controls (4.17,-0.84) and (1.99,-0.18) .. (0,0) .. controls (1.99,0.18) and (4.17,0.84) .. (6.56,1.97)   ;
%Shape: Ellipse [id:dp8039206204196873] 
\draw  [draw opacity=0][fill={rgb, 255:red, 255; green, 0; blue, 33 }  ,fill opacity=1 ][dash pattern={on 4.5pt off 4.5pt}] (208.03,201.4) .. controls (208.03,199.79) and (209.33,198.48) .. (210.94,198.48) .. controls (212.54,198.48) and (213.84,199.79) .. (213.84,201.4) .. controls (213.84,203.02) and (212.54,204.32) .. (210.94,204.32) .. controls (209.33,204.32) and (208.03,203.02) .. (208.03,201.4) -- cycle ;
%Curve Lines [id:da08105047674061427] 
\draw    (146.57,223.43) .. controls (149.83,234.97) and (149.9,239.65) .. (165.27,248.5) ;
\draw [shift={(167,249.48)}, rotate = 209.05] [color={rgb, 255:red, 0; green, 0; blue, 0 }  ][line width=0.75]    (4.37,-1.32) .. controls (2.78,-0.56) and (1.32,-0.12) .. (0,0) .. controls (1.32,0.12) and (2.78,0.56) .. (4.37,1.32)   ;
\draw [shift={(146,221.48)}, rotate = 73.42] [color={rgb, 255:red, 0; green, 0; blue, 0 }  ][line width=0.75]    (4.37,-1.32) .. controls (2.78,-0.56) and (1.32,-0.12) .. (0,0) .. controls (1.32,0.12) and (2.78,0.56) .. (4.37,1.32)   ;

% Text Node
\draw (173,180) node [anchor=north west][inner sep=0.75pt]  [font=\scriptsize] [align=left] {A};
% Text Node
\draw (224,52) node [anchor=north west][inner sep=0.75pt]  [font=\scriptsize] [align=left] {B};
% Text Node
\draw (213,185.6) node [anchor=north west][inner sep=0.75pt]  [font=\scriptsize] [align=left] {C};
% Text Node
\draw (245,237.6) node [anchor=north west][inner sep=0.75pt]  [font=\scriptsize] [align=left] {D};
% Text Node
\draw (131,239.4) node [anchor=north west][inner sep=0.75pt]  [color={rgb, 255:red, 255; green, 0; blue, 35 }  ,opacity=1 ]  {$2\beta $};


\end{tikzpicture}

  \caption{The evolution of polarisation along the closed loop, whose area gives us the geometric phase.}
  \label{loop}
\end{figure}
\noindent
When a light beam's polarisation state is transformed along a closed loop as shown in Fig. \ref{loop} on the surface of the Poincar\'e sphere, it acquires a geometric phase shift $\Phi$, related to the solid angle $\Omega$ subtended by the loop at the centre of the sphere: 
$$\Phi = -\frac{1}{2}\Omega$$
\subsection{Experimental Setup}
We measure this geometric phase through a modified Michelson interferometer setup as shown below.
\begin{figure}[H]
  \centering
  

% Pattern Info
 
\tikzset{
pattern size/.store in=\mcSize, 
pattern size = 5pt,
pattern thickness/.store in=\mcThickness, 
pattern thickness = 0.3pt,
pattern radius/.store in=\mcRadius, 
pattern radius = 1pt}
\makeatletter
\pgfutil@ifundefined{pgf@pattern@name@_awtfzjn5t}{
\pgfdeclarepatternformonly[\mcThickness,\mcSize]{_awtfzjn5t}
{\pgfqpoint{0pt}{0pt}}
{\pgfpoint{\mcSize+\mcThickness}{\mcSize+\mcThickness}}
{\pgfpoint{\mcSize}{\mcSize}}
{
\pgfsetcolor{\tikz@pattern@color}
\pgfsetlinewidth{\mcThickness}
\pgfpathmoveto{\pgfqpoint{0pt}{0pt}}
\pgfpathlineto{\pgfpoint{\mcSize+\mcThickness}{\mcSize+\mcThickness}}
\pgfusepath{stroke}
}}
\makeatother

% Pattern Info
 
\tikzset{
pattern size/.store in=\mcSize, 
pattern size = 5pt,
pattern thickness/.store in=\mcThickness, 
pattern thickness = 0.3pt,
pattern radius/.store in=\mcRadius, 
pattern radius = 1pt}
\makeatletter
\pgfutil@ifundefined{pgf@pattern@name@_81sh3ry7e}{
\pgfdeclarepatternformonly[\mcThickness,\mcSize]{_81sh3ry7e}
{\pgfqpoint{0pt}{0pt}}
{\pgfpoint{\mcSize+\mcThickness}{\mcSize+\mcThickness}}
{\pgfpoint{\mcSize}{\mcSize}}
{
\pgfsetcolor{\tikz@pattern@color}
\pgfsetlinewidth{\mcThickness}
\pgfpathmoveto{\pgfqpoint{0pt}{0pt}}
\pgfpathlineto{\pgfpoint{\mcSize+\mcThickness}{\mcSize+\mcThickness}}
\pgfusepath{stroke}
}}
\makeatother
\tikzset{every picture/.style={line width=0.75pt}} %set default line width to 0.75pt        

\begin{tikzpicture}[x=0.75pt,y=0.75pt,yscale=-1,xscale=1]
%uncomment if require: \path (0,244); %set diagram left start at 0, and has height of 244

%Shape: Rectangle [id:dp14185075563007155] 
\draw   (249,100.38) -- (256,100.38) -- (256,118) -- (249,118) -- cycle ;
%Straight Lines [id:da11999295642892993] 
\draw    (249,100.38) -- (256,118) ;
%Shape: Rectangle [id:dp9184167694561414] 
\draw  [fill={rgb, 255:red, 128; green, 128; blue, 128 }  ,fill opacity=1 ] (304,101.38) -- (311,101.38) -- (289,119) -- (282,119) -- cycle ;
%Shape: Rectangle [id:dp3226822044129056] 
\draw  [pattern=_awtfzjn5t,pattern size=3.75pt,pattern thickness=0.75pt,pattern radius=0pt, pattern color={rgb, 255:red, 0; green, 0; blue, 0}] (398,98.48) -- (406,98.48) -- (406,122.48) -- (398,122.48) -- cycle ;
%Shape: Rectangle [id:dp3922700766631825] 
\draw  [pattern=_81sh3ry7e,pattern size=3.75pt,pattern thickness=0.75pt,pattern radius=0pt, pattern color={rgb, 255:red, 0; green, 0; blue, 0}] (307,13.48) -- (307,21.48) -- (283,21.48) -- (283,13.48) -- cycle ;
%Shape: Rectangle [id:dp8691680750804505] 
\draw  [fill={rgb, 255:red, 180; green, 176; blue, 176 }  ,fill opacity=0.55 ] (259,212.95) -- (335,212.95) -- (335,225.95) -- (259,225.95) -- cycle ;
%Shape: Path Data [id:dp24287901299796077] 
\draw  [fill={rgb, 255:red, 255; green, 34; blue, 34 }  ,fill opacity=1 ] (166,115) -- (166,103.95) -- (193.91,103.95) -- (203,109.48) -- (193.91,115) -- (166,115) -- cycle ;
%Straight Lines [id:da7951766541334498] 
\draw    (295,20.48) -- (297,213.12) ;
%Shape: Ellipse [id:dp054621829572002456] 
\draw  [fill={rgb, 255:red, 80; green, 227; blue, 194 }  ,fill opacity=0.78 ] (282,158.34) .. controls (282,155.77) and (288.49,153.68) .. (296.5,153.68) .. controls (304.51,153.68) and (311,155.77) .. (311,158.34) .. controls (311,160.91) and (304.51,163) .. (296.5,163) .. controls (288.49,163) and (282,160.91) .. (282,158.34) -- cycle ;
%Straight Lines [id:da3891888243730185] 
\draw    (201,109.48) -- (398,110.48) ;
%Shape: Rectangle [id:dp46802720018590027] 
\draw  [fill={rgb, 255:red, 248; green, 231; blue, 28 }  ,fill opacity=0.51 ] (358,103.48) -- (375,103.48) -- (375,115.48) -- (358,115.48) -- cycle ;
%Shape: Rectangle [id:dp16215682794788222] 
\draw  [fill={rgb, 255:red, 255; green, 193; blue, 195 }  ,fill opacity=0.67 ] (329,104.48) -- (344,104.48) -- (344,114.48) -- (329,114.48) -- cycle ;
%Straight Lines [id:da2328044363426376] 
\draw    (336,104.38) -- (336,114.93) ;
%Straight Lines [id:da7850503412568411] 
\draw    (366.5,103.48) -- (366.5,116.03) ;
\draw   (269,106) -- (275,109.68) -- (269,113.35) ;
\draw   (300,55.68) -- (295.66,62.35) -- (291.32,55.68) ;
\draw   (321,114.35) -- (314.32,110.01) -- (321,105.68) ;
\draw   (291.32,43.35) -- (295.66,36.68) -- (300,43.35) ;
\draw   (301,136.68) -- (296.66,143.35) -- (292.32,136.68) ;
\draw   (383,107) -- (389,110.68) -- (383,114.35) ;
%Shape: Wave [id:dp5073112018846967] 
\draw  [color={rgb, 255:red, 255; green, 0; blue, 31 }  ,draw opacity=1 ] (259,201.63) .. controls (259.78,205.13) and (260.53,208.35) .. (261.4,208.35) .. controls (262.3,208.35) and (263.08,204.85) .. (263.9,201.18) .. controls (264.72,197.5) and (265.5,194) .. (266.4,194) .. controls (267.3,194) and (268.08,197.5) .. (268.9,201.18) .. controls (269.72,204.85) and (270.5,208.35) .. (271.4,208.35) .. controls (272.3,208.35) and (273.08,204.85) .. (273.9,201.18) .. controls (274.72,197.5) and (275.5,194) .. (276.4,194) .. controls (277.3,194) and (278.08,197.5) .. (278.9,201.18) .. controls (279.72,204.85) and (280.5,208.35) .. (281.4,208.35) .. controls (282.3,208.35) and (283.08,204.85) .. (283.9,201.18) .. controls (284.72,197.5) and (285.5,194) .. (286.4,194) .. controls (287.3,194) and (288.08,197.5) .. (288.9,201.18) .. controls (289.72,204.85) and (290.5,208.35) .. (291.4,208.35) .. controls (292.3,208.35) and (293.08,204.85) .. (293.9,201.18) .. controls (294.72,197.5) and (295.5,194) .. (296.4,194) .. controls (297.3,194) and (298.08,197.5) .. (298.9,201.18) .. controls (299.72,204.85) and (300.5,208.35) .. (301.4,208.35) .. controls (302.3,208.35) and (303.08,204.85) .. (303.9,201.18) .. controls (304.72,197.5) and (305.5,194) .. (306.4,194) .. controls (307.3,194) and (308.08,197.5) .. (308.9,201.18) .. controls (309.72,204.85) and (310.5,208.35) .. (311.4,208.35) .. controls (312.3,208.35) and (313.08,204.85) .. (313.9,201.18) .. controls (314.72,197.5) and (315.5,194) .. (316.4,194) .. controls (317.3,194) and (318.08,197.5) .. (318.9,201.18) .. controls (319.72,204.85) and (320.5,208.35) .. (321.4,208.35) .. controls (322.3,208.35) and (323.08,204.85) .. (323.9,201.18) .. controls (324.72,197.5) and (325.5,194) .. (326.4,194) .. controls (327.3,194) and (328.08,197.5) .. (328.9,201.18) .. controls (329.72,204.85) and (330.5,208.35) .. (331.4,208.35) .. controls (332.3,208.35) and (333.08,204.85) .. (333.9,201.18) .. controls (334.59,198.05) and (335.26,195.06) .. (336,194.23) ;

% Text Node
\draw (324,88) node [anchor=north west][inner sep=0.75pt]  [font=\scriptsize] [align=left] {QP1};
% Text Node
\draw (356,88) node [anchor=north west][inner sep=0.75pt]  [font=\scriptsize] [align=left] {QP2};
% Text Node
\draw (319,153) node [anchor=north west][inner sep=0.75pt]  [font=\scriptsize] [align=left] {Lens};
% Text Node
\draw (379,130) node [anchor=north west][inner sep=0.75pt]  [font=\scriptsize] [align=left] {Mirror M\textsubscript{2}};
% Text Node
\draw (317,12) node [anchor=north west][inner sep=0.75pt]  [font=\scriptsize] [align=left] {Mirror M\textsubscript{1}};
% Text Node
\draw (279,214) node [anchor=north west][inner sep=0.75pt]  [font=\scriptsize] [align=left] {Screen};
% Text Node
\draw (229,124) node [anchor=north west][inner sep=0.75pt]  [font=\scriptsize] [align=left] {Polariser};
% Text Node
\draw (308,37.4) node [anchor=north west][inner sep=0.75pt]  [font=\scriptsize]  {$1$};
% Text Node
\draw (348,120.4) node [anchor=north west][inner sep=0.75pt]  [font=\scriptsize]  {$2$};
% Text Node
\draw (312,136.4) node [anchor=north west][inner sep=0.75pt]  [font=\scriptsize]  {$3$};
% Text Node
\draw (165,121) node [anchor=north west][inner sep=0.75pt]  [font=\scriptsize] [align=left] {Laser};
% Text Node
\draw (298,117.8) node [anchor=north west][inner sep=0.75pt]  [font=\tiny] [align=left] {Beam\\Splitter};


\end{tikzpicture}

  \caption{Schematic diagram of the experimental setup}
\end{figure}
\noindent
The laser source is passed through a polariser to obtain a pure linearly polarised light which passes through a beam-splitter, creating a coherent superposition. The ray in arm 1 goes and reflects back from mirror $\mathrm{M}\tund{1}$. The ray in arm 2 passes through two quarter waveplates.\\[0.2cm]
On passing through $\mathrm{QP}\tund{1}$, the linearly polarised light becomes right circularly polarised (path $\mathrm{AB}$ in Fig. \ref{loop}). The second waveplate $\mathrm{QP}\tund{2}$ is rotated by an angle $\beta$ with the original polarization, which changes the right circularly polarised light backe to linear polarisation but introduces a phase (path $\mathrm{BC}$ in Fig. \ref{loop}). On reflecting from mirror $\mathrm{M}\tund{2}$, the evolution is similar but in an opposite sense (path $\mathrm{CD}$ and $\mathrm{DA}$ in Fig. \ref{loop}). Thus, a closed loop is traversed in the Poincar\'e sphere.\\[0.2cm]
The ray comes back to the beam-splitter, with linear polarisation but with an additional phase. The two rays recombine and we obtain an interference pattern on the screen. \\[0.2cm]
Throughout the experiment, we keep the mirrors fixed, hence the \textit{dynamical phase} is kept constant. Hence, the fringe patterns due to interference is solely controlled by the \textit{geometric phase}. Below, we show the mathematical calculation for the emergence of the geometric phase. 
\subsection{Analysis using Jones matrices}
The incident light from the laser, after passing through the polariser, can be denoted by the Jones vector 
\begin{equation}
  \vb{E}\tund{in} = \mqty(1\\0)
\end{equation}
This light will pass through the waveplates. The Jones matrix for a quarter waveplate with its axis at an angle $\theta$ is given as,
\begin{equation}
  \vb{J}\tund{Q}(\theta) =\mqty(\cos^2\theta + i\sin^2\theta & (1-i)\sin\theta\cos\theta\\  (1-i)\sin\theta\cos\theta & \sin^2\theta + i\cos^2\theta)
\end{equation}
Thus, on passing through $\mathrm{QP}\tund{1}$ the state becomes,
\begin{equation}
  \vb{E}\tund{1} = \vb{J}\tund{Q}(45\degree)  \mqty(1\\0) \equiv \frac{1}{\sqrt{2}}\mqty(1 & -i \\ -i & 1)\mqty(1\\ 0) =\frac{1}{\sqrt{2}} \mqty(1 \\ -i)
\end{equation}
On passing through $\mathrm{QP}\tund{2}$ the state becomes,
\begin{equation}
  \vb{E}\tund{2} = \vb{J}\tund{Q}(+\beta)\frac{1}{\sqrt{2}}\mqty(1 \\ -i) = e^{i\beta} \frac{1}{\sqrt{2}}\mqty(\cos\beta + \sin\beta\\ \cos\beta - \sin\beta)
\end{equation}
Now, on reflecting from the mirror $\mathrm{M}\tund{2}$ the state becomes, 
\begin{equation}
  \vb{E}\tund{3} =e^{i\beta} \mqty(1& 0 \\ 0& -1)\frac{1}{\sqrt{2}}\mqty(\cos\beta + \sin\beta\\ \cos\beta - \sin\beta) = e^{i\beta}\frac{1}{\sqrt{2}}\mqty(\cos\beta + \sin\beta\\  \sin\beta-\cos\beta)
\end{equation}
Again, after passing through $\mathrm{QP}\tund{2}$ we get,
\begin{equation}
   \vb{E}\tund{4} = \vb{J}\tund{Q}(-\beta)e^{i\beta}\frac{1}{\sqrt{2}}\mqty(\cos\beta + \sin\beta\\  \sin\beta-\cos\beta) = e^{2i\beta}\frac{1}{\sqrt{2}}\mqty(1 \\ -i)
\end{equation}
Finally, on passing through $\mathrm{QP}\tund{1}$ we have,
\begin{equation}
  \vb{E}\tund{5} = \vb{J}\tund{Q} (-45\degree)e^{2i\beta}\frac{1}{\sqrt{2}}\mqty(1 \\ -i) =e^{2i\beta} \frac{1}{2}\mqty(1 & i \\ i & 1)\mqty(1\\ -i) = e^{2i\beta}\mqty(1\\ 0)
\end{equation}
We see that we obtain the initial polarisation state but with an additional phase of $e^{2i\beta}$, $$\vb{E}\tund{out} = e^{-2i\beta}\vb{E}\tund{in}$$
\section{Data and Calculations}
After alignment of the Michelson interferometer, we varied the angle $\beta$ of the second quarter waveplate and cumulatively counted the number of fringe shift occuring with the varying angle. The data is presented in the table below: 
\begin{table}[H]
\centering
\caption{Measured Phase Shift vs. Rotation Angle $\beta$}
\label{tab:phase_shift}
\begin{tabular}{cccc}
\toprule
Angle $\beta$ ($\degree$) & Angle $\beta$ (rad) & Fringe Shift ($n$) & Phase $\Delta\Phi = 2\pi n$ (rad) \\
\midrule
168  & 2.93  & 1  & 06.28  \\
248  & 4.33  & 2  & 12.57 \\
520  & 9.08  & 3  & 18.85 \\
740  & 12.92 & 4  & 25.13 \\
898  & 15.67 & 5  & 31.42 \\
1074 & 18.74 & 6  & 37.70 \\
1312 & 22.90 & 7  & 43.98 \\
1492 & 26.04 & 8  & 50.27 \\
1612 & 28.13 & 9  & 56.55 \\
1906 & 33.27 & 10 & 62.83 \\
1986 & 34.66 & 11 & 69.12 \\
2180 & 38.05 & 12 & 75.40 \\
2360 & 41.19 & 13 & 81.68 \\
2489 & 43.44 & 14 & 87.96 \\
\bottomrule
\end{tabular}
\end{table}
\noindent
The plot between the rotation angle $\beta$ and the phase shift of the fringe pattern $\Delta \Phi = 2\pi n$ is shown below. 
\begin{figure}[H]
  \centering 
  \includegraphics[scale=0.97]{../beta_vs_fringe.pdf}
\end{figure}
\noindent
From the linear fit of the data points, we obtained a slope of $m= 1.945 \pm 0.033$ which is close to the theoretically expected value of $m\tund{th}=2.0$ (since the geometric phase is $2\beta$, $\Delta \Phi = 2\beta$)
\section{Error Analysis}
The percentage error from the theoretical value is, 
$$\boxed{\mathrm{Error}. (\%) = \frac{(2-1.945)}{2}\times 100\% = 2.75\%}$$
We also calculate the propagation of error. 
\subsection{Sources of Error}
\begin{itemize}
\item The fringes were perturbed by minor vibrations, which made it difficult to record the shift. 
\item Imperfect alignment of the optical components led to reduced fringe visibility and distorted fringe patterns, which occasionally made it difficult to identify fringe shifts accurately.
\item Impurities in the optical elements produced an undesirable shadow on the screen on top of the fringes which hindered the recording of the fringes. 
\end{itemize}
\section{Discussion and Conclusion}
In this experiment, we observed the Pancharatnam phase through the interference pattern using a Michelson interferometer, through the use of waveplates.\\[0.2cm]
We observed a clear shift in the interference fringes as the second quarter-wave plate was rotated. By plotting the accumulated phase shift against the rotation angle $\beta$, we obtained an approximately linear relation with a slope $ 1.945\pm 0.033$ close to the theoretical value of $2$. This confirms that the observed phase is mainly geometric in nature and depends only on the change in polarization, not on the optical path length, since we kept the optical path length fixed during the experiment.\\[0.2cm]
As a further study, additional optical elements can be introduced to modify the polarization evolution, and the resulting fringe shifts can be compared with theoretical predictions obtained using Jones matrix calculations.\\[0.2cm]
The geometric phase provides a robust means of controlling phase, as it is largely insensitive to fluctuations in optical path length, making it useful for various optical applications. Through this experiment, we demonstrated an important and fundamental aspect of light arising purely from its polarisation evolution.
\end{document}


